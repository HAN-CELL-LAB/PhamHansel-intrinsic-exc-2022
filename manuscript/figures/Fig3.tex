\begin{figure}[ht]
    \centering
    \includegraphics[width=0.9\textwidth,center]{figures/Fig3.pdf}
    \caption{
    \textit{Description of the abstract model with binary output units and the task of concern}.
    (\textbf{a}) Summary of model and task, and parameters involved.
    The model is a feedforward network comprised of an input layer ($X$ - 150 units in total)
        and output layer ($Y$ - 3 units in total).
    The colors represent the selectivity for a specific output unit in each input unit.
    (\textbf{b}) The excitatory connectivity matrix $W_{YX}$ is characterized by the selectivity index $\alpha_W$ and the number of overlap.
    Each sub-panel is an example with different pairs of ($\alpha_W$, \# overlap).
        Higher $\alpha_W$ (from left to right) concentrates more input weights on the corresponding output that inputs are selective for.
        Higher number of overlap (top to bottom) creates more inputs selective for all output units, and simultaneously more inputs without any selectivity.
    (\textbf{c})
        The output activation function is a Heaviside-step function characterized by
            the constant base threshold ($\theta_{\mathrm{base}}$ for all output units,
            and a certain threshold change $\Delta\theta$ for only one output unit
            (in this example the first output $Y_1$, colored red).
        The input patterns are parameterized by the ``\% complete'' for each pattern -- higher \% complete (top to bottom) recruits more inputs.
        There is Gaussian noise added on top of the input patterns, with a fixed $\sigma_\mathrm{noise}$ (not shown here for simplicity).
    }
    \label{fig:demo-ffwd-recog}
\end{figure}