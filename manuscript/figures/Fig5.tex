\begin{figure}[H]
    \centering
    \includegraphics[width=1.2\textwidth,center]{figures/Fig5.pdf}
    \caption{
    \textit{Threshold potentials and plasticity: analysis of biological data sets}.
    (\textbf{a}) Biological data of
            neural spiking threshold ($V_T$),
            resting potentials ($V_R$)
            and their differences  ($\Delta V_{TR} = V_T - V_R$)
        from \textit{in vivo} V1 recordings
            obtained from \cite{Li2020-ej}.
        In each violin plot,
            horizontal lines show the mean,
            while white circles show
                the first quartile, median and third quartile from bottom to top.
    (\textbf{b}) Biological data of $V_T$, $V_R$ and $\Delta V_{TR}$
        from \textit{in vitro} data taken from \cite{Gill2020-wy}
        at initial values ($V^0$)
        and after plasticity induction ($V^f$).
        Colors indicate different plasticity induction protocols:
            \textit{electrical} activation (somatic depolarization or synaptic activation) in blue;
            \textit{cholinergic} group using oxo-m bath application in green;
            and \textit{cholinergic paired} with electrical activation (somatic depolarization) in red.
    (\textbf{c}) The recognition tradeoff (shades of red)
            and recognition true-positive rate (shades of blue)
            as a function of threshold change ($\Delta\theta$)
            for three conditions of selectivity indices ($\alpha_W$)
        are shown as an example of the model calculations
            to compare with the lower and upper threshold potential bounds measured in mouse S1 cortex.
        The lower bound is the average of resting potentials ($n = 23$),
            while the upper bound is the average of threshold potentials
                during the relative refractory period calculated from a new data set
                ($n = 4$, see \autoref{supp:demo-refrac-thres}).
    (\textbf{d}) Relationship between functional change (number of spikes $n_{spk}$)
            and membrane potential change ($V_R, V_T, \Delta V_{TR}$)
            after induction protocols are applied (color legends are same as in panel b).
        Change is denoted as
            $x^{\Delta} = x^{\mathrm{f}} - x^{\mathrm{0}}$,
            as the difference between initial and final values of variable $x$.
        Colored lines are linear fits using data from each induction group,
            while dashed gray lines are linear fits using all the data.
    (\textbf{e})
    Distribution of S1 recordings
        from the three different experimental groups
        according to category membership,
        based on $\Delta V_{TR}^{\Delta}$,
        to either
        ``negative threshold shift''
        or ``non-negative threshold shift''.
        See text and \autoref{supp:s1-data-dV_TR} for more information.
    }
    \label{fig:ffwd-biol-data}
\end{figure}

