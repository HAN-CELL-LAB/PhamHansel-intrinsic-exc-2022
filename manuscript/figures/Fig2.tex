
\begin{figure}[ht]
    \centering
    \includegraphics[width=1.2\textwidth,center]{figures/Fig2.pdf}
    \caption{
    \textit{Dependence of the number of input combinations resulting in supra-threshold activation ($N = 10$) on the threshold $\theta$}.
    Supra-threshold activation of the output $Y$ means that
        the weights sum of the inputs $X$ is greater than the threshold $\theta$,
        in other words $Y = 1$ if $\sum_i W_i x_i > \theta$.
    (\textbf{a}) and (\textbf{b}) are examples of the weights drawn from
        a uniform distribution (a)
        and when all are assigned the same value (b), respectively,
        before normalization to the sum.
    In each example panel:
        (\textit{i}) shows the sorted input weights from $X \to Y$;
        (\textit{ii}) shows the sorted weighted sum of the corresponding combination
        of the corresponding combinations of the inputs (\textit{iii});
        (\textit{iv}) shows whether corresponding combinations in (\textit{ii,iii}) would result in supra-threshold activation as a function of the threshold $\theta$ (x-axis);
        (\textit{v}) indicates the number of combinations in (\textit{iv}) showing supra-threshold activation, as a function of the threshold $\theta$ (x-axis).
    (\textbf{c}) Mean sorted weights for 50 instantiations for
        the equal-weight scenario (dotted line)
        and for the scenario where weights are initially drawn from a uniform distribution (dashed line).
    (\textbf{d}) Mean number of combinations resulting in supra-threshold activation, as a function of the threshold.
    }
    \label{fig:demo-simple}
\end{figure}