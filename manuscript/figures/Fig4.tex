\begin{figure}[!ht]
    \centering
    \includegraphics[width=0.85\textwidth,center]{figures/Fig4.pdf}
    \caption{
    \textit{Recognition performance based on the change of threshold in a single output unit}.
    (\textbf{a})
        The recognition trade off (TPR $-$ FPR; shades of red)
        and recognition true-positive rate (TPR; shades of blue)
        as a function of threshold change ($\Delta\theta$).
        Increasing selectivity indices ($\alpha_W$) are shown from left to right.
        Two conditions of selectivity overlap are shown from top to bottom.
        Darker shades represent more complete patterns.
    \NEWCHANGE{(\textbf{b})
        Receiver operating characteristics (ROC) of recognition
        with TPR and FPR plotted against each other
        with shadings indicating input completeness
        for $\alpha_W = 0.59$.
    }
    (\NEWCHANGE{\textbf{c}})
        Demonstration of how $\Delta \theta_{\mathrm{optim}}$ is obtained,
        as the threshold change with the smallest absolute value to achieve maximal performance.
    (\NEWCHANGE{\textbf{d}})
        $\Delta \theta_{\mathrm{optim}}$ of recognition TPR (\textbf{c}$_1$)
        and tradeoff (TPR $-$ FPR, \textbf{c}$_2$)
        as a function of input completeness (y-axis)
        and selectivity setup (strength/index: grey shadings; overlap: top vs bottom).
    }
    \label{fig:ffwd-recog-perf}
\end{figure}
