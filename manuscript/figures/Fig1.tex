
\begin{figure}[ht]
    \centering
    \includegraphics[width=0.9\textwidth,center]{figures/Fig1.pdf}
    \caption{
    \textit{Output threshold decrease reduces discriminability}.
    (\textbf{a}) Schematic of the model and quantification of discriminability.
        For each $k$, a set of binary inputs $\{X^{(i)}\}$ with the first set of $k$ active units
            (minimal overlapping activation) are considered more ``discriminable''
        if the pairwise distance (either with Jaccard distances $J_d$ or normalized Hamming distances $H_d$)
            of the corresponding outputs $\{Y^{(i)}\}$ is larger.
        $W$ is the weight matrix between $X$ and $Y$, $\theta$ is the output layer threshold
            while $\Delta \theta$ is the applied threshold change.
    (\textbf{b}) Illustration of the way $J_d$ (red) and $H_d$ (blue) change for specific pairs of outputs
        when $\Delta \theta$ changes the excitability of the outputs (green dots).
    (\textbf{c}, \textbf{d}) The mean $J_d$ (red) and $H_d$ (blue) distances and mean output activity (black) are plotted
        as a function of $\Delta \theta$ for different $k$ (shadings)
        of $N_X = 10$ input units and different number of output units $N_Y$.
    }
    \label{fig:discrim}
\end{figure}