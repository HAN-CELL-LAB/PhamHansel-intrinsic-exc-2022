\section{Discussion}

The first main finding from our study is
        that intrinsic plasticity benefits and enables neural coding strategies
            that aim at the recognition of known, but incompletely presented input patterns.
    This interpretative coding comes at the expense of coding precision;
        this effect is a direct consequence of the loss of resolution
            that results from the fact that intrinsic amplification applies
            to all synaptic inputs located ``upstream'' of where the excitability is changed.
    As such, this phenomenon is not restricted to near-somatic changes in spike threshold,
        but will equally apply to excitability changes specific to dendritic compartments
        \citep{Losonczy2008-pi,Ohtsuki2012-ds}.

The second main finding is that a shift in coding mode
        is promoted by electrical activity paired with cholinergic signaling,
        but not by electric or cholinergic activation alone.
    This observation is of interest, because of what the coding modes imply.
    High-accuracy representation and encoding of input patterns is advantageous
        when unknown patterns need to be distinguished.
    It is not known what the pattern should look like,
        so an incomplete pattern cannot be distinguished from the complete, unknown pattern.
    Interpretative encoding, in contrast,
        matches a pattern against a known pattern,
        and reports whether
            the presented pattern belongs to an expected category
            or matches a specific object of interest.
    The finding that this shift in encoding strategy results
        from paired cholinergic and electric activation is of interest,
        as this activation configuration can be expected to create -- via learning --
        known patterns out of previously unknown ones
        upon repeated presentation and with a certain level of attention.
    Our identification of cholinergic co-activation as a critical contributor to IP
        does not exclude other, additional mechanisms.
    For example, it has been shown that IP in cortical pyramidal neurons
        is triggered by the activation of metabotropic glutamate receptors (mGluRs) that
            regulate small-conductance, calcium-dependent SK-type K channels \citep{Sourdet2003-qr}.
    Thus, the cellular machinery is present to allow for excitability alterations by a variety of mechanisms.
    \textit{In vivo} recordings from cortical pyramidal neurons confirm, however, that
        the ultimate control of bidirectional IP and threshold plasticity is executed by ion channels
            that can indeed affect excitability parameters,
            namely primarily Na$^+$ and/or K$^+$ conductances \citep{Mahon2012-bt}.

In an initial step, we demonstrate that increased excitability
        -- in form of a reduction in threshold potential --
        harms output discrimination.
    This is because intrinsic excitability amplifies responsiveness of $>$ 1 synaptic input,
        which results in enhanced ``merging'' of output patterns,
        and  a reduction in the resolution of object / pattern representation.
    In contrast, performance in a recognition task does improve with reduced spike thresholds,
        in particular under conditions that are challenging for perception,
        such as a low degree of input selectivity in the output cells $Y$ and/or the presentation of incomplete patterns.
    The latter observation is a positive consequence of the unspecific nature of amplification:
        a reduction in the spike threshold will allow a higher number of input combinations (synapses) to evoke spike firing.
    To stay in our example of neurons specialized in face recognition,
        successful detection of a face (as an object category)
        or successful detection of a specific, individual face
        becomes possible even when details of the object are missing or only details are presented.
    Such flexibility in input detection is one of the requirements for the detection of known objects,
        which points to an essential nature of intrinsic plasticity mechanisms in neuronal encoding strategies.
    In expansion of such role, it has been observed that neurons fire
        at similar rates for pairs of objects,
            for whom an associative relationship has been learned,
        while they fire at different rates for pairs of objects
            that have no learned association \citep{Freedman2011-bo}.
    This finding suggests that membrane excitability
        -- in addition to shared synaptic drive --
        can become a signature of encoded objects that the network groups into object categories.
    Synaptic connectivity remains an important factor, and therefore it is noteworthy
        that a similar amplification would result from an increase of weights across all excitatory synapses.
    The reason why biological networks do not seem to use this synaptic strategy is possibly
        that such upregulation would depend on the activity of all synapses.
    It would be nearly impossible to do so without changing synaptic weight ratios,
        with unforeseeable consequences for object representation.

The improvement in performance of the recognition task upon lowering of the spike threshold is limited
        by the fact that false-positives emerge next to true-positives.
    The FPR rate will increase alongside the TPR rate when intrinsic amplification rises,
        and thus the probability of incorrect object identification (false recognition) increases as well.
    The difference between these detection rates, TPR $-$ FPR, depends on
        the selectivity index $\alpha_W$ (TPR $-$ FPR increases with high selectivity values),
        the degree of input overlap (it increases with low overlap)
        and pattern completeness (it increases with a high degree of completeness).
    The dependence on these parameters allows us to determine optimal spike thresholds
        for different parameter constellations,
        with maximal relative changes of $-$ 0.4 to $-$ 0.5 being required under non-preferential detection conditions.
    While the occurrence of false-positives cancels out
            some of the advantages of intrinsic plasticity in the recognition of objects,
        it is possible that biological networks may compensate for this effect by
            identifying false-positives via network effects.
    In this scenario, individual neurons may signal positive recognition at short-delays
        (and this coding would be reflected in the TPR rate),
        while computations within the larger network would enable correction for false-positives;
        in such case, the TPR rate would actually better reflect performance than the TPR $-$ FPR difference.

\subsubsection*{Reference to biological excitability data}

We have analyzed data sets obtained from
        whole-cell patch-clamp recordings from L2/3 pyramidal cells in V1 cortex of awake macaques \citep{Li2020-ej}
        and in S1 cortex of mice \textit{in vitro} \citep{Gill2020-wy}.
    These data sets make it possible to assess
        the range of assumed thresholds under awake conditions (macaques)
        and the plasticity range under \textit{in vitro} conditions (mice).
    The recordings from cortical slices that are used here include
        activation protocols that trigger intrinsic plasticity.
    Our analysis shows that in V1 cortex the threshold is up to $20$ mV more positive than the resting potential.
    In S1 cortex, the distance is somewhat higher, up to $30$ mV.
    This difference in $\Delta V_{TR}$ might well be partially related to the species and experimental conditions.
    However, unlike the primary somatosensory cortex \citep{Brecht2003-vf},
        visual cortices are known to operate in a ``high-input regime'' that is signified by
            strong synaptic input
            and high-frequency spike activity at rest \citep{Shadlen1998-je}.
    It therefore is possible that these different types of neurons
        adjust their spike thresholds differently
        to optimize their firing properties for different neural coding goals.

We find that all three activation patterns
        -- electric, cholinergic and paired --
        all tend to increase excitability as measured by the number of evoked spikes.
    In contrast, the threshold potential is bidirectionally changed over a total range of about $10$ mV.
    Negative threshold changes are almost exclusively initiated
        by paired electric and cholinergic activation (\autoref{fig:ffwd-biol-data}d,e).
    As the spike threshold is the only excitability parameter
        that is informative about spiking vs non-spiking conditions,
        the threshold is critical in determining a neuron's ability to participate in any neuronal ensemble.


\subsubsection*{Realistic constraints for abstract model variables}

The hyperparameters used
        (selectivity strength, overlap and input incompleteness)
        are somewhat abstract
        and do not in themselves carry information about relevant parameter ranges.
    Therefore, we offer below some additional notes
        and predictions on values that these hyperparameters may realistically assume.

First, imperfect \textit{selectivity strength} ($\alpha_W < 1$)
        is typically expected in realistic settings
        as individual neurons are known to respond to considerable input variance.
    While the realistic range of $\alpha_W$ values is difficult to determine,
        we can make some simple predictions regarding the relationship between
            selectivity strength and intrinsic excitability.
    Increased object familiarity,
        via repeated presentation of a certain input,
        would allow $\alpha_W$ to increase.
    In fact, repeated synaptic activity
        -- which conveys the information that an input pattern is repeatedly or persistently present --
        may lead to co-occurring synaptic and intrinsic potentiation \citep[e.g.][]{Belmeguenai2010-ee}.
    This is meaningful as $\alpha_W$ is likely to stay $ < 1$
        (a value of 1 would be equivalent to the presence of only one input,
        albeit potentially represented by multiple synapses).
    Our model shows that $\alpha_W$ -- over a wide range of assumed values $ < 1$ --
        makes it necessary to reduce threshold to maximize task performance.
    Thus, the co-occurrence of synaptic potentiation with enhanced excitability has the capacity
        to move both parameters into an optimal parameter space.

Second, nonzero \textit{selectivity overlap} in lower layers of encoding
        is expected in biological systems,
        especially for distributed coding of complex objects.
    Such objects often are composed of building blocks
        that are on their own similar between otherwise different objects,
        as much as a Lego figure is different from any other,
        but the individual Lego pieces are the same.
    A biological example is a face that is entirely unique,
        even though specific components -- such as eye color --
            only exist in a limited number of variations.
    This phenomenon causes overlap
            as eyes are a common input component to encode for faces;
        even with consideration of eye colors,
            the same eye color might be shared between otherwise different faces.
    The presence of overlap is not always a limiting factor.
    For instance, it could allow for
        (a) reinforced recognition, especially in weak input conditions,
        or (b) contribution of encoding object co-occurrence for higher-order concepts.
    However, at the same time, high selectivity overlap
        often poses a challenge for differentiation
            (in discrimination and recognition tasks),
        which generally is made easier the more object components differ.

Third, while selectivity overlap is a potential intrinsic network concern
        and is sometimes difficult to quantify,
        \textit{input incompleteness} (or sparsity) is intuitively relevant
            as a concern in real-world settings.
    How much this parameter matters for recognition, we argue,
        depends on the ``need'' (goal) for such recognition task,
        which might further constrain the optimal excitability states.
    For example, engrams responsible for predator detection might benefit from
        higher excitability with some leeway on false positives
        (for a field mouse,
            recognition of a predator by only a few simple features,
                such as recognition of a bird of prey by its silhouette,
            enhances the probability of survival).
    In face recognition cells in primates,
        which may recognize an individual in a position-independent manner,
        input completeness is expected to always be $< 1$,
        because the object can only be presented in one position at a time,
        leaving the other possible object constellations unrepresented.
    Such parameter constraints determine critical ranges
        for optimal threshold values that our model describes.

\subsubsection*{Hints from artificial settings in machine learning}

Machine learning algorithms show astounding analogies to IP.
    For example, batch-normalization (BN) is a form of input normalization
        that is applied during the training of deep learning models.
    Though the exact mechanism is still debated, BN is known to speed up training
            and to smooth the optimization landscape \citep{Ioffe2015-bw,Santurkar2018-ff};
        in addition, it serves as a form of regularization during training \citep{Luo2018-vw}.
    A recent study \citep{Shaw2020-hu} points out
        (a) the similarity between neural IP and machine learning BN
            as both perform certain transformations of the inputs
            with adaptive parameters based on the history of the input statistics,
        and (b) that their implementation speeds up training
            in image classification tasks
            trained on multi-layer-perceptron and convolutional networks.
    Interestingly, another study \citep{Frankle2020-qs} shows that
        deep artificial neural networks trained solely using BN parameters
        could reach nontrivial classification performance.
    Combining both of these recent works, one could hypothesize that
        if synaptic weights of a neural network are frozen
            or the weight learning rate is much slower than the intrinsic update
            (which is biologically plausible, for example
                \cite{Lopez-Rojas2016-ew} shows dendritic IP is more easily inducible than SP),
        appropriate changes to the parameters characterizing the artificial nonlinearity
            could achieve significant improvement with fewer adaptive parameters.
    In other words, given a ``status-quo'' network topology,
        training only the intrinsic parameters to adapt artificial single unit nonlinearity
        might achieve better performance in an energy efficient manner,
            which is advantageous when it is expensive to change the synaptic profiles of the network \citep{Gallistel2017-em}.
    In both biological and machine learning scenarios, it appears that
        IP enables a shift from faithful / accurate to interpretative decoding,
            which may be an essential coding strategy in fast-changing environments
                where some object features remain hidden,
        but objects nevertheless need to be swiftly recognized, or at least categorized.

The work presented here highlights the role that
        threshold plasticity in particular plays in this adaptation of coding strategy.
    It further shows that cholinergic co-activation,
        a cellular modulation mechanism associated with attention
        and thus detection of signal salience,
    promotes the adaptation step.
