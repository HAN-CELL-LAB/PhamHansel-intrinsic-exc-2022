\subsection{Recognition of known / learned input patterns in a binary two-layer feedforward network}

\subsubsection*{Overview description}

We use a simple feedforward network with two layers:
        150 input units in the $X$ layer
        and 3 binary output units in the $Y$ layer (\autoref{fig:demo-ffwd-recog}a).
    Each output unit is considered to represent a input pattern.
    The weight matrix $W_{YX}$ is non-negative, characterized by
        a selectivity index $\alpha_W$
        and selectivity overlap (\autoref{fig:demo-ffwd-recog}b).

The inputs are generated at $X$, characterized by
        a \% completeness to allow for incomplete input patterns,
        with additive normal-distributed noise (\autoref{fig:demo-ffwd-recog}c).
    The outputs at $Y$ are binary,
        activated by a Heaviside step function with
            their own thresholds set to a base threshold $\theta_{\mathrm{base}}$
            and an additional change $\Delta \theta$ applied to
            \NEWCHANGE{the threshold of only one output neuron}
            (\autoref{fig:demo-ffwd-recog}a).
    \NEWCHANGE{Otherwise,
        all neurons in the input layer $X$
        and in the output layer $Y$, respectively,
        are identical (non-parameterized)}.
    The task of interest is \textit{recognition} of a known / learned pattern of interest by the tested output unit.
    The term ``incomplete pattern'' in the context of the recognition task refers to a deviation from the known / learned pattern,
        in which components of the complete pattern are missing.

\NEWCHANGE{Note that our model does not contain recurrent network components}
    \NEWCITE{\citep{Hopfield1982-ax}.}
\NEWCHANGE{This selection was made as our goal is
    to isolate emerging consequences of cell-autonomous intrinsic plasticity of neurons,
        whose output function informs a measure of task performance.
    It is anticipated that
        with the recurrent connectivity properties of a Hopfield model
            (recurrent connectivity within the input layer),
        output layer activation would differ.
    However, we expect that the consequences of threshold changes
        in an output layer neuron for task performance
        would qualitatively remain the same.
}


\subsubsection*{Selectivity and connectivity}

The 150 inputs in $X$ are selective for either none, only one or all of the outputs in $Y$.
    The number of inputs selective to each output unit is kept fixed at
        $N_{\text{selec}}=N_X/N_Y=150/3 = 50$.

The non-negative weights in matrix $W_{YX}$ are characterized by the \textit{selectivity index}
    $\alpha_W \in [0,1]$ (\autoref{eq:ffwdweight-alphaW})
    -- a higher value means that more weights are concentrated on the corresponding selective output (\autoref{fig:demo-ffwd-recog}b, left to right).
    Thus, $\alpha_W$ describes selectivity in responsiveness of the outputs in $Y$.

\vspace{1em}
\begin{equation}
    \label{eq:ffwdweight-alphaW}
    \begin{aligned}
        W_{Y_j X} &= \left\{
                W_{\mathrm{selec}},
                W_{\mathrm{nonselec}}
                \right\} \\
            \text{ where } &
            \begin{cases}
                W_{\mathrm{selec}} &=
                    \alpha_W
                    \cdot
                    \unfnormgen{N_{\text{selec}}} \\
                W_{\mathrm{nonselec}} &=
                    (1 - \alpha_W)
                    \cdot
                    \unfnormgen{N_X - N_{\text{selec}}}
            \end{cases}
    \end{aligned}
\end{equation}


Additionally, the weight matrix is also characterized by \textit{selectivity overlap}
    (overlap in the inputs in $X$).
    For example, \autoref{fig:demo-ffwd-recog}b top panels are examples of zero selectivity overlap,
        while the bottom panels have 20 input units that overlap and are represented by all 3 outputs.
    For each parameter set, we run 10 instantiations of $W_{YX}$.

The identity of which output(s) each input is selective for is defined beforehand,
    and is also influenced by the selectivity overlap parameter
    but the number of selective inputs per output remains fixed ($N_{\text{selec}}=50$).

\subsubsection*{Input generation}

The inputs at $X$ are generated with binary states (before noise is added)
        to represent only one pattern in each example.
    To represent input incompleteness, the parameter \textit{\% complete}
        (\autoref{fig:demo-ffwd-recog}c)
        controls the percentage of the inputs that are active.
    On top of this, there is an additive noise
        drawn from a zero-mean normal distribution
        with a fixed width of $\sigma_{\text{noise}} = 0.8$.

\subsubsection*{Output activation}

The output units at $Y$ are activated by a Heaviside step function (\autoref{eq:heavisidefun}),
    each with their own threshold value.
    For simplicity, we set all output thresholds
        to a base threshold $\theta_{\text{base}} = 0.5$,
        and apply a change only to one output unit with an amplitude of $\Delta \theta \in [-0.5,0.5]$.

\subsubsection*{Task of interest}

For each manipulation of $\Delta \theta$ in a selected output unit,
        we generate 2000 random input patterns
        and characterize the outputs at $Y$ to evaluate performance of \textit{recognition} (of known / learned patterns),
        via the true-positive rate (TPR) of \textit{only} the output whose threshold is changed with $\Delta \theta$.
    Activity in other output units is disregarded
        (in other words, all the other outputs would be considered noise
        \NEWCHANGE{or irrelevant; note that if threshold changes in multiple neurons were considered, the error rate is expected to differ}).
    Receiver operating characteristics
        (\NEWCHANGE{ROC; a method to show classification performance, graphically displayed as TPR vs FPR})
        analysis for recognition is plotted in
        \NEWCHANGE{{\autoref{fig:ffwd-recog-perf}}b}.

\subsubsection*{Optimal activation threshold change}

For each parameter set and performance curve $f(\Delta \theta)$ of a given task,
    $\Delta \theta_{\text{optim}}$ is defined as the change of threshold $\Delta \theta$
        whose absolute value is the smallest,
        while resulting in the highest performance outcome.
    In other words, we define the \textit{``optimal'' threshold change}
        as the change nearest to 0 to achieve the maximum performance.
    These are shown in \autoref{fig:ffwd-recog-perf}
        \NEWCHANGE{c,d}.

\subsection{Measurement of biological resting and threshold potentials}

\subsubsection*{V1 \textit{in vivo} biological thresholds}

We obtained the \textit{in vivo} resting and threshold potentials
        of layer 2/3 neurons in V1 from recordings performed in awake macaques \citep{Li2020-ej}
        (see summary of data in \autoref{fig:ffwd-biol-data}a).
    We denote membrane thresholds as $V_T$,
        resting potentials as $V_R$
        and the distance between them as $\Delta V_{TR} = V_T - V_R$.
    The biological data statistics are reported as (mean $\pm$ SEM).

\subsubsection*{S1 \textit{in vitro} biological thresholds}

We calculate resting and threshold potentials from \textit{in vitro} recordings
        from mouse S1 L2/3 pyramidal neurons that we published previously \citep{Gill2020-wy}.
    This data set includes three groups of intrinsic plasticity experiments:
        (i) \textit{electric group} -- electric tetanization using somatic depolarization or synaptic tetanization,
        (ii) \textit{cholinergic group} -- bath application of the agonist of muscarinic acetylcholine receptors (mAChRs) oxotremorine-m (oxo-m),
        and (iii) \textit{cholinergic paired group} -- co-application of oxo-m when electric stimulation is applied.
    \NEWCHANGE{Bath application of an mAChR agonist is intended
        to activate mAChRs that are primarily located
        on the extrasynaptic membrane of pyramidal cell dendrites and spines}
        \NEWCITE{\citep{McCormick1985-la, Yamasaki2010-di}.}
    The somatic depolarization protocol in (i) consisted of
        10 depolarizing current pulses (50 ms duration) that
        were delivered at 10 Hz, followed by 2 seconds of holding current.
    The depolarization amplitude was adjusted to evoke one to three spikes during each of these pulses
        and was repeated 100 times for a total duration of 5 minutes.
    The synaptic tetanization protocol in (i) mimicked the overall timing presented in the somatic protocol,
        but now somatic current injection was replaced by
            extracellular stimulation using 10 Hz pulses (applied for 1 second)
            followed by 2 seconds of no stimulation.
    This pattern was repeated 100 times.
    In the co-activation protocol (iii), only somatic depolarization was used.
    For further experimental details, see \cite{Gill2020-wy}.
    \NEWCHANGE{Six new recordings were added
        to increase the statistical power of the analysis presented here.
        These recordings followed the same protocol as described in
    } \NEWCITE{\cite{Gill2020-wy}}.

To calculate the resting potentials $V_R$, we use
        the holding current ($I_{\mathrm{hold}}$)
        and the recorded potential 20 ms before test pulse ($V_{\mathrm{spont}}$),
        along with the input resistance ($R_{\mathrm{in}}$):
            $V_R = V_{\mathrm{spont}} - I_{\mathrm{hold}} R_{\mathrm{in}}$
            (assuming zero net active components at rest at steady state).
    The threshold potential $V_T$ is detected
        as the potential first crossing 8 $V/s$
        \citep[for a similar approach, see][]{Mahon2012-bt, Popescu2021-cj}
        around 1 ms before the \textit{first} action potential.
    We also look at the difference between resting and threshold potentials
        $\Delta V_{TR} = V_T - V_R$ to assess intrinsic excitability.
    The distributions of these data are shown in \autoref{fig:ffwd-biol-data}b.

For \textit{baseline} (pre-induction; denote with superscript $x^{\mathrm{0}}$ for a property $x$),
        we take the average measures (either resting or threshold potentials)
        of the last 4 sweeps before application of any plasticity protocol.
    For \textit{post} measurements (denote as $x^{\mathrm{f}}$),
        we take at least 10 sweeps near the end of the experiment of each cell for average.
    Change between pre- and post-induction properties is denoted as
        $x^{\mathrm{\Delta}} = x^{\mathrm{f}} - x^{\mathrm{0}}$
        (\autoref{fig:ffwd-biol-data}d).
    Additionally in this figure panel, linear fits, along with their p-values and $R^2$
        are shown for functional excitability changes
        (number of spikes $n_{spk}^{\Delta}$)
        related to changes in membrane potential measures
        ($V_R^{\Delta}, V_T^{\Delta}, \Delta V_{TR}^{\Delta}$).

In addition, we determined the theoretically conceivable lower and upper threshold bounds
        which, for illustrative purposes, are shown together
        with our modeling results in \autoref{fig:ffwd-biol-data}c.
    We define the biological lower bound
        as the average of all resting potentials in \cite{Gill2020-wy}
        \NEWCHANGE{, plus new experiments}.
    To define the biological threshold upper bound,
        we obtained recordings from 4 additional S1 cells
        \citep[same recording conditions as in][]{Gill2020-wy}.
    In these recordings, we evoked multiple action potentials
        by injecting depolarizing current pulses.
    We took the threshold potentials measured
        during the relative refractory state
        as the upper bound of threshold potentials.
    This relative refractory state is characterized
        by a reduction in AP amplitude after the first spike was fired
        as well as a spike threshold at a more depolarized membrane potential.


\subsubsection{Statistical analysis}

\NEWCHANGE{
    Statistical significance was determined by using
        the two-sample unpaired Student's $t$ test
            (between-group comparison;
            V1 vs S1 cortex)
        and the paired Student's $t$ test
            (within-group comparison of baseline vs post-manipulation changes;
            plasticity experiments),
        when appropriate.
    The Pearson correlation coefficient was used
        to statistically describe correlations between
            spike output
            and threshold potential (and related parameters).
    All statistics are reported as mean $\pm$ SD.
    The chi-square test is used to examine
        the distribution of negative vs non-negative threshold shifts
        in the plasticity experiments.
}


