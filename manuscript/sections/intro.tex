\section{Introduction}

Synaptic plasticity, and in particular long-term potentiation \citep[LTP;][]{Bliss1973-od,Nabavi2014-fn},
    is widely considered as the primary cellular correlate of learning.
    The synaptic theory of learning has been challenged, however,
        based on the observation that properties of behavioral learning,
        such as the duration of the learning phase or learned time intervals,
        do not match properties of synaptic plasticity \citep[see][]{Gallistel2013-qm,Gallistel2014-el}.
    Cell-autonomous forms of plasticity, intrinsic to activated cells,
        emerge as alternative learning correlates \citep{Gershman2021-eu} that may
            replace \citep{Johansson2015-ql,Gallistel2017-em,Johansson2019-sg}
            or critically complement \citep[e.g.][]{Titley2017-da,Hansel2020-ep,Josselyn2020-et}
        synaptic plasticity in its role in learning.
    Cell-autonomous plasticity may rest on prolonged changes in membrane excitability and spike firing properties \citep{Marder1996-mp,Hansel2001-ti},
        and this type of plasticity (``intrinsic plasticity''; IP) is considered here.
    Note, however, that cell-autonomous plasticity is not restricted to changes in excitability,
       and that excitability changes -- when they occur -- can also be homeostatic in nature \citep{Turrigiano1999-pn}.
    IP has been observed
        in response to activity patterns or in the context of behavioral learning both \textit{in vitro} and \textit{in vivo},
        in types of neurons as diverse as cortical pyramidal neurons \citep[e.g.][]{Paz2009-yn,Mahon2012-bt,Moyer1996-bq}
        and cerebellar Purkinje cells \citep[e.g][]{Schreurs1998-de,Belmeguenai2010-ee,Titley2020-df}.

Computationally, the role of IP (including homeostatic IP) may be to
        adjust the membrane threshold and excitability to an optimal target rate
            \citep{Lazar2007-fl, Naude2013-bt, Zheng2014-op, Li2018-ev, Del_Papa2019-du,Loidolt2020-pw,Wu2020-ru},
        or it may be to alter the excitability set-point to maximize information transfer between synaptic inputs and spike output
            \citep{Stemmler1999-iz, Triesch2005-jo, Triesch2007-np, Joshi2009-yy, Savin2010-bf, Li2013-qk,Li2016-ge, Zhang2019-eo, Zhang2019-qh, Shaw2020-hu}.
    IP models have been used to explain
        homeostasis \citep{LeMasson1993-fr,Marder2002-lv,Wu2020-ru},
        sequence learning \citep{Loidolt2020-pw},
        networks with rich dynamics \citep{Naude2013-bt, Zheng2014-op},
        criticality of neural networks \citep{Naude2013-bt,Del_Papa2019-du}
        as well as receptive field development \citep{Triesch2005-jo, Triesch2007-np,Savin2010-bf}.
    At the same time, their application has been explored in
        reservoir computing for classification and regression tasks \citep{Steil2007-ql,Wang2020-us},
        as well as more recently in machine learning algorithms \citep{Zhang2019-eo, Zhang2019-qh, Shaw2020-hu}
        and neuromorphic computing implementation \citep{Dalgaty2019-ng, Baek2020-si}.
    In contrast to these implications, the role of IP in the encoding of input patterns
        -- e.g. in object recognition in sensory physiology,
        but also in more abstract pattern encoding in general --
        remains underexplored in computational biology.
    This is despite the fact that threshold setting is a well-known problem (and solution)
        in the engineering of software for face/object recognition.

In neural networks, the issue at hand is generally the same,
        but presents itself in a slightly more complex way as individual neurons
        -- via their synaptic input --
        may receive information about several unrelated objects / patterns.
        Moreover, network connectivity is not determined by engineering goals,
            but is set by biological rules during brain development.
\NEWCHANGE{While principles of cortical connectome development and architecture are known}
    \NEWCITE{\citep[e.g][]{Innocenti1981-gi, Felleman1991-az, Innocenti1995-ge,Singer1995-wi, Izhikevich2008-po},}
    \NEWCHANGE{it is difficult to determine the settings of specific network parameters, such as
        synaptic weights,
        input/output selectivity,
        or neuronal excitability}.
    A phenomenon that is shared, though, is that a threshold reduction at the output stage
        (in biological neurons: IP being triggered at or near the soma) facilitates activation,
        but may reduce the resolution of input representation, thus enhancing ``graininess''.
    In a biological neuron, enhanced excitability via IP results in an amplification
        that does not differentiate between synaptic inputs ``upstream'' of the location
            that experienced the increase in excitability.
    The consequences of this phenomenon for perception and signal encoding in neural networks
        have not been studied so far.
    Hence, we set out to investigate the consequences of IP
        for the detection and differentiation of unknown input patterns
            as well as the recognition of known / learned patterns.
    There are different physiological aspects of IP,
        including changes in the spike threshold,
        in excitability as measured by the number of evoked spikes,
        or the amplitude of the afterhyperpolarization (AHP) that may follow spike bursts.
    Our study focuses on the spike threshold as the only IP parameter
        that by definition separates ``non-spike'' from ``spike'' states,
        which determine whether information encoded by a neuron is further conveyed through the network.

Here the term ``pattern'' is used to describe general input patterns,
        which may be any pattern of information that is conveyed via synapses onto neurons.
    However, for illustrative purposes we will use examples from object recognition in sensory physiology.
    Indeed, the inspiration for the recognition task comes from a simplified analogy to face recognition
        -- of both individual faces and ``face'' as an object category --
        posed by \cite{Titley2017-da} when discussing the problem of neuronal activation and integration into ensembles,
        in situations where fine-scale object details are missing,
            or only local structural inputs are presented.
    With the neural code (portfolio of information that is represented)
        of concept cells set up by synaptic connectivity patterns,
            enhancing intrinsic excitability through IP would
            amplify the ultimate responsiveness of the neuron to all its inputs without differentiation.
    This could result in a more robust representation
        (and possibly categorization) of the concept (e.g. a face) of interest,
        when its inputs lack certain details.
    Moreover, such effects of categorization by design can be obtained by
        regulating intrinsic amplification alone,
        without the need for changing synaptic weights,
        which might come at the cost of changing synaptic weight ratios
        and thus possibly the information content that is represented.

Here we perform a parametric study on a two-layer feedforward network
        consisting of inputs selective for certain binary output units.
    Our main goal is to determine consequences of threshold change for network performance in two tasks:
        a) a pattern distinction task, in which IP is expected to lower performance
        and b) a recognition task for reporting already known / learned patterns.
    In contrast to similar computations used in software engineering applications,
        our study introduces ambiguities inherent to biological or biologically inspired artificial neural networks
            that arise from unknown connectivity parameters:
                selectivity of the output units (neurons) for input features,
                and selectivity in the presentation of such features by the input units.
    The main result of our study is that a decrease in output threshold may facilitate pattern recognition,
        in particular when inputs are incomplete and network selectivity is imperfect.
    This finding demonstrates that IP -- in form of a negative threshold shift --
        is beneficial under conditions when neurons have to interpret incomplete input patterns and assign them to a specific pattern or object category.
    In addition, we re-analyze published data sets obtained
        during whole-cell patch-clamp recordings from L2/3 pyramidal neurons in cortex
        to determine physiological threshold ranges.
    We identify electrical activation paired with cholinergic signaling
        as an activity pattern that
        promotes a \NEWCHANGE{lasting} negative threshold change
        and thus drives neurons into a signaling mode that favors the detection of known input patterns.
