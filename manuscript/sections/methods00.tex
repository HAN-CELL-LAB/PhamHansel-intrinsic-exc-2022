\subsection{Discrimination of input patterns in a two-layer feedforward network}

The model used consists of a two-layer feedforward network
        with $N_X$ (usually 10) number of input $X$ units
        and $N_Y$ number of binary output $Y$ units.
    They are connected via the weight matrix $W$,
        first initialized from a uniform distribution $\unfdistrib{0}{1}$,
        then normalized to a sum of 1 for each $Y$ (L1 normalization).
    In other words, the entries in $W$ are constructed as
        $W_{Y_j,X} = \unfnormgen{N_X} \ \forall \ j \in [1, N_Y]$
        (\autoref{eq:defunifnorm}).

\vspace{-1em}
\begin{align}
    \unfnormgen{N} = \left\{
            \dfrac{w_i}{\sum_i w_i}
            \right\}
        \text{ where } w_i \sim \unfdistrib{0}{1}
        \text{ for }
        i \in [1, N]
    \label{eq:defunifnorm}
\end{align}

The output activity $Y_j$ is binary,
        characterized by a Heaviside activation function,
        with a threshold $\theta_j$ (\autoref{eq:heavisidefun}).
    Decreased thresholds are synonymous with increased neural intrinsic excitability.
    Initially, these thresholds are sampled from $\unfdistrib{0}{1}$.

\vspace{-1em}
\begin{align}
    Y_j = H(W_{Y_j X} X | \theta_j) =
    1 \text{  if  } \sum_i W_{ji} X_i > \theta_j
    \text{ or } 0 \text{  otherwise}
    \label{eq:heavisidefun}
\end{align}

When presenting all possible binary input patterns,
        we quantify the entropy of the output patterns (\autoref{eq:entropyfun}) as a function of $W, \theta$,
        along with the number of unique output patterns
        to present a measure of discriminability at the output layer.
    We also optimize such measure at baseline (i.e. $\Delta \theta = 0$)
        and then observe the effects of output threshold changes on performance in the discrimination task.


\vspace{-1em}
\begin{align}
    S(Y) = - \sum_j
        P\left(\sigma_Y^{(j)}\right)
        \log_2 P\left(\sigma_Y^{(j)}\right)
    \text{ where } P\left(\sigma_Y^{(j)}\right)
    \text{ is the probability of the output state }
    \sigma_Y^{(j)} \in \{0,1\}^{N_Y}
    \label{eq:entropyfun}
\end{align}


We choose a simple approach to optimize the Euclidean distance with a randomly selected output pattern set.
For each network instantiation,
    a random mapping between $2^{N_X}$ input patterns ($\sigma_X^{(j)} \in \{0,1\}^{N_X}$)
        and output patterns ($\sigma_{Y^\prime}^{(j)} \in \{0,1\}^{N_Y}$)
    is chosen to optimize for the Euclidean distance loss (\autoref{eq:euclideanloss}).
We track the entropy of the output patterns along the process
    and pick the maximum entropy state as our ``best'' solution.
    This approach was selected over other loss function choices, because of its inherent simplicity.

\vspace{-1em}
\begin{align}
    L(Y, Y^{\prime}) = \left \langle \lVert Y - Y^{\prime} \rVert_2 \right \rangle
    \text{ where } Y^{\prime} \text{ is the ``desired'' output state}
    \text{ and } \lVert \bullet \rVert_2 \text{ is the L2 norm}
    \label{eq:euclideanloss}
\end{align}

In addition, for a specific minimally shared input pattern $k$
        (for simplicity, assumed for only the first $k$ input units that are always activated),
        we quantify the resulting mean pairwise output distances
        (either Jaccard distances $J_d$ in \autoref{eq:jaccarddist}
        or normalized Hamming distances $H_d$ in \autoref{eq:hammingdist})
        as another measure of discriminability.
    This is shown in \autoref{fig:discrim} for the best states,
        with $N_X=10$ for different $N_Y$.

\vspace{-1em}
\begin{align}
    J_d(Y^{(1)}, Y^{(2)}) &= 1 -
        \frac{
            \# \left\{
                i \Big| Y^{(1)}_i = Y^{(2)}_i = 1
                \right\}
        }{
            \# \left\{
                j \Big| Y^{(1)}_j = 1 \text{ or } Y^{(2)}_j = 1
                \right\}
        }
        \text{ where }
        \begin{dcases}
            Y^{(1)}, Y^{(2)} \text{ are output vectors}
            \\
            \# \left\{\bullet\right\}
            \text{ is the cardinality of a given set}
        \end{dcases}
    \label{eq:jaccarddist}
    \\
    H_d(Y^{(1)}, Y^{(2)}) &= 1 -
        \frac{
            \# \left\{
                i \Big| Y^{(1)}_i = Y^{(2)}_i
                \right\}
        }{
            N_Y
        }
    \label{eq:hammingdist}
\end{align}


Both quantify the number of differences between binary patterns,
        but $J_d$ is normalized by the total number of activated locations between both patterns,
        while $H_d$ is normalized by the length of the patterns.
    Note that this results in $J_d$ being more biased towards activated units,
        and not being defined when there are no activated units in both patterns
        (hence the red lines are cut off at the right hand-side of the panels in \autoref{fig:discrim}).
    These simulations were run on the Neuroscience Gateway \citep{Sivagnanam2013-au}.
