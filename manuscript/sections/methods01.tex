\subsection{Model to test the effect of threshold changes on the number of inputs needed for suprathreshold activation}

To demonstrate that incomplete pattern recognition could benefit from increased excitability,
        we use a simple binary unit $Y$ receiving inputs from 10 binary $X$ units.
    For simplicity, we represent the connectivity in the network with an excitatory matrix $W$ (\autoref{fig:demo-simple}a,b top).
    The weights are constructed either from
        a uniform distribution $\unfdistrib{0}{1}$ (\autoref{fig:demo-simple}a)
        or set equal (\autoref{fig:demo-simple}b),
        then normalized to a sum of 1.
    The normalization allows us to control the range of the output threshold.
    In other words, $W = \unfnormgen{N_X = 10}$ (\autoref{eq:defunifnorm}).

The output activity $Y$ is binary, characterized by a Heaviside activation function,
        with a threshold $\theta \in [0, 1]$ (\autoref{eq:heavisidefun}).
    We assume that $Y$ represents an input pattern (such as a complex object or concept)
        that the downstream network associates with activation of all of these input units.
    Hence, \textit{incomplete patterns} would be any binary representation of $X$
        where at least one unit is $0$ (\autoref{fig:demo-simple}a-\textit{iv}, b-\textit{iv}).
