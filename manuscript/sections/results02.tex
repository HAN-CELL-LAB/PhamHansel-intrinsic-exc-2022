\subsection{Recognition of known / learned patterns in a simple feedforward two-layer binary network}

\begin{figure}[ht]
    \centering
    \includegraphics[width=0.9\textwidth,center]{figures/Fig3.pdf}
    \caption{
    \textit{Description of the abstract model with binary output units and the task of concern}.
    (\textbf{a}) Summary of model and task, and parameters involved.
    The model is a feedforward network comprised of an input layer ($X$ - 150 units in total)
        and output layer ($Y$ - 3 units in total).
    The colors represent the selectivity for a specific output unit in each input unit.
    (\textbf{b}) The excitatory connectivity matrix $W_{YX}$ is characterized by the selectivity index $\alpha_W$ and the number of overlap.
    Each sub-panel is an example with different pairs of ($\alpha_W$, \# overlap).
        Higher $\alpha_W$ (from left to right) concentrates more input weights on the corresponding output that inputs are selective for.
        Higher number of overlap (top to bottom) creates more inputs selective for all output units, and simultaneously more inputs without any selectivity.
    (\textbf{c})
        The output activation function is a Heaviside-step function characterized by
            the constant base threshold ($\theta_{\mathrm{base}}$ for all output units,
            and a certain threshold change $\Delta\theta$ for only one output unit
            (in this example the first output $Y_1$, colored red).
        The input patterns are parameterized by the ``\% complete'' for each pattern -- higher \% complete (top to bottom) recruits more inputs.
        There is Gaussian noise added on top of the input patterns, with a fixed $\sigma_\mathrm{noise}$ (not shown here for simplicity).
    }
    \label{fig:demo-ffwd-recog}
\end{figure}

To extend the framework, we set up our model as a simple feedforward network
    with an input layer $X$ and an output layer $Y$,
    connected via the non-negative connectivity matrix $W_{YX}$
        that describes the amount of selectivity of the inputs to each output unit via
            the selectivity index $W$ (selectivity of the output unit $Y$)
            and a specified amount of input overlap (overlap in pattern presentation in the input unit $X$)
            (\autoref{eq:ffwdweight-alphaW}, \autoref{fig:demo-ffwd-recog}a,b).

The (in)completeness of the inputs is characterized by
        the \% complete parameter, with added noise.
    The output units at $Y$ are activated with a Heaviside step function (\autoref{eq:heavisidefun}), whose thresholds are characterized by
        the base thresholds $\theta_{\text{base}}$
        and an additive threshold change $\Delta \theta$ in only one of the output units (\autoref{fig:demo-ffwd-recog}c).
    $\Delta \theta < 0$ would be analogous to an increased intrinsic excitability of the affected artificial neuron compared to the other output units,
        whereas $\Delta \theta > 0$ would indicate decreased intrinsic excitability.
    Here, we look at the effects of $\Delta \theta$ on the performance in a recognition task
        (focus on the output whose threshold is changed) (\autoref{fig:demo-ffwd-recog}c).

The results are shown in \autoref{fig:ffwd-recog-perf}a,
        recognition true positive rate (TPR) is shown in blue,
        while the tradeoff between TPR and the false-positive rate, FPR, (TPR-FPR) is shown in red.
    Because $N_Y=3$, $\alpha_W < 1/3$ would be lower than chance,
        and thus highly undesirable for recognition.
    Increased selectivity emphasizes relevant inputs and reduces the weight of irrelevant ones, thus benefitting both performance parameters.

\begin{figure}[!ht]
    \centering
    \includegraphics[width=0.85\textwidth,center]{figures/Fig4.pdf}
    \caption{
    \textit{Recognition performance based on the change of threshold in a single output unit}.
    (\textbf{a})
        The recognition trade off (TPR $-$ FPR; shades of red)
        and recognition true-positive rate (TPR; shades of blue)
        as a function of threshold change ($\Delta\theta$).
        Increasing selectivity indices ($\alpha_W$) are shown from left to right.
        Two conditions of selectivity overlap are shown from top to bottom.
        Darker shades represent more complete patterns.
    (\textbf{b})
        Demonstration of how $\Delta \theta_{\mathrm{optim}}$ is obtained,
        as the threshold change with the smallest absolute value to achieve maximal performance.
    (\textbf{c})
        $\Delta \theta_{\mathrm{optim}}$ of recognition TPR (\textbf{c}$_1$)
        and tradeoff (TPR $-$ FPR, \textbf{c}$_2$)
        as a function of input completeness (y-axis)
        and selectivity setup (strength/index: grey shadings; overlap: top vs bottom).
    }
    \label{fig:ffwd-recog-perf}
\end{figure}


The dependence of recognition performance on $\Delta\theta$ (true positive rate; TPR)
    appears sigmoidal, inverted with the sign of threshold change
        (blue shaded lines in \autoref{fig:ffwd-recog-perf}a)
        \TOREMOVE{{
            \autoref{supp:ffwd-recog-more}},
            {\autoref{supp:ffwd-recog-tradeoff-roc}}a}.
    As expected, lowered thresholds allow for easier recognition of the output of interest, eventually reaching 100\% TPR.
    When the selectivity index is high and the inputs are mostly complete, the unit does not require decreased thresholds to reach maximal recognition.
    However, for a given input completeness, weaker selectivity requires a more pronounced threshold decrease to reach a set performance level.
    The same is true when inputs become more incomplete, given a weak to moderate $\alpha_W$.
    Hence the optimal threshold change for maximal recognition leans toward stronger intrinsic potentiation $\Delta \theta < 0$
        when either the inputs become more insufficient or the network selectivity strength $\alpha_W$ is imperfect (\autoref{fig:ffwd-recog-perf}
        \NEWCHANGE{c,d}).

\NEWCHANGEUNSURE{However}\footnote{might need to remove this entire paragraph},
    in situations of high TPR recognition resulting from a strong threshold decrease,
        the false-positive rate (FPR) will also be high,
        since the affected output will be active even when the inputs come from other input patterns.
    Inspecting the FPR curves,
        \TOREMOVE{(
            {\autoref{supp:ffwd-recog-more}},
            {\autoref{supp:ffwd-recog-tradeoff-roc}}a)}
        \footnote{unclear how to keep this whole paragraph without referencing a supp figure},
        we observe increases of FPR monotonically with decreases of thresholds.
    With $\alpha_W < 1/3$ (first column in panel a), the weights distribute more strongly for irrelevant inputs than for relevant inputs
        (\autoref{fig:ffwd-recog-perf}a, left-most column),
        hence a strong threshold decrease would lead to recognition of the other outputs more readily than the output of interest,
        generally causing higher false positives than true positives
        (i.e. FPR $>$ TPR
        \TOREMOVE{, see also {\autoref{supp:ffwd-recog-tradeoff-roc}}b where FPR and TPR are plotted against each other}).
    Decreased $\alpha_W$ worsens the outcome and leads to higher FPR
        as the affected output will be more easily evoked from strong irrelevant inputs.
    With stronger selectivity index (higher $\alpha_W$) without overlap,
        input completeness has less of an effect on FPR changes compared to TPR
        \TOREMOVE{(compare the top right sub-panels in {\autoref{supp:ffwd-recog-tradeoff-roc}}a)}.
    Overall, increased input incompleteness leads to decreases of FPR
        as irrelevant inputs will have less of a chance to evoke responses in the relevant output unit $Y$.

As the goal is to maximize recognition performance without enhancing the false recognition rate,
        we want to maximize TPR while minimizing FPR,
        thus maximizing the difference TPR $-$ FPR
        (red shaded lines in \autoref{fig:ffwd-recog-perf}a).
    When selectivity is too low
        ($ < 1/3$, leftmost top sub-panel in \autoref{fig:ffwd-recog-perf}a
        and lightest grey line in the top right sub-panel in
        \autoref{fig:ffwd-recog-perf}\NEWCHANGE{d}$_2$),
        FPR is higher than TPR as irrelevant input strengths are now stronger than relevant connections,
        hence TPR $-$ FPR $<$ 0.
    In moderate selectivity strengths, these difference curves are negatively affected by increased input incompleteness
        and require more negative $\Delta \theta_{\mathrm{optim}}$ to reach their maximum
        (\autoref{fig:ffwd-recog-perf}\NEWCHANGE{d}).

\subsubsection*{Effects of input selectivity overlap}

Aside from the imperfection of selectivity ($\alpha_W < 1$),
        another possible issue that could arise intrinsically in the feedforward setup is
        an overlap in input selectivity, as certain input units could activate more than one output.
    Here, we consider the worst case, in which some units connect strongly to all outputs (\autoref{fig:demo-ffwd-recog}b).
    As we assume a fixed selective input number per output ($N_{\text{selec}} = 50$),
        nonzero overlap input units send relevant,
        but also irrelevant input to the output units.
    When selectivity strength is low, this reduces the relative contribution of relevant input.

As defined, recognition TPR is not really affected by the selectivity overlap,
        as we focus only on output units encoding the learned object pattern
        (\autoref{fig:ffwd-recog-perf}a,
        \TOREMOVE{{\autoref{supp:ffwd-recog-more}}, {\autoref{supp:ffwd-recog-tradeoff-roc}}a}).
    \TOREMOVE{The only effect that selectivity overlap has on recognition would be increased FPR
        ({\autoref{supp:ffwd-recog-more}},
        {\autoref{supp:ffwd-recog-tradeoff-roc}}),
        especially with more complete inputs and an increased selectivity index.}
    This is because the high overlap allows the shared inputs
        to evoke responses in the affected output too easily
        when the input pattern actually belongs to another output class.
    For this reason, sparsity of input patterns is more beneficial
        for the tradeoff between TPR and FPR (i.e. the difference) when $\Delta \theta < 0$.
    On the other hand, when $\Delta \theta > 0$, input completeness is better for the recognition tradeoff.
    However, if we look at the optimal threshold change,
        as input patterns become more incomplete and have a moderate selectivity index,
        $\Delta \theta_{\mathrm{optim}}$ generally tends towards more negative changes
        in the case of nonzero selectivity overlap
        (\autoref{fig:ffwd-recog-perf}\NEWCHANGE{d}, bottom sub-panel).

A well-known analysis tool for examining binary classification performance is the ROC graph.
    As expected,
        \TOREMOVE{either}
        increased input completeness
        \TOREMOVE{or selectivity strength}
        would increase recognition performance
        \NEWCHANGE{{\autoref{fig:ffwd-recog-perf}b}}
        \TOREMOVE{for sufficient $\alpha_W$ ($ > 1/3$)
            as shown in {\autoref{supp:ffwd-recog-tradeoff-roc}}b
            and by examining the increase of the AUC of the ROC as a function of either parameter
            (see {\autoref{supp:ffwd-recog-ROCAUC}})}.
    Increase in selectivity overlap decreases the overall performance of the recognition task
        (compare top and bottom panels of
        \NEWCHANGE{{\autoref{fig:ffwd-recog-perf}b})}
        \TOREMOVE{{\autoref{supp:ffwd-recog-tradeoff-roc}}b,
            which is further exemplified by
            a slower increase of AUC of as function of $\alpha_W$
            or input completeness ({\autoref{supp:ffwd-recog-ROCAUC}})}.

\subsection{Biological threshold potentials and activity-dependent plasticity}

The purpose of re-analyzing existing biological data sets is twofold.
    First, we would like to obtain information on the distribution of threshold values
        and the difference between resting and threshold potentials in cortex.
    Second, as our modeling has shown, negative changes in threshold potentials
        can signify a shift in neural encoding strategies
        from accurate reporting of input patterns
        to interpretative reporting of whether a known / previously learned pattern is presented.
    In this latter scenario, it is of interest,
        which biological activity pattern drives this threshold adaptation
        and thus enables optimal signal processing
        when the history of the input pattern presentation (previously presented under favorable conditions or not) changes.

\subsubsection*{Threshold potentials obtained from V1 \textit{in vivo} recordings}

We obtained resting and threshold potentials
        from published whole-cell patch-clamp recordings
        from L2/3 pyramidal neurons in the primary visual cortex (V1)
        of awake macaque monkeys \citep{Li2020-ej}.
    These recordings were performed
        in the laboratory of Dr. Nicholas Priebe (University of Texas at Austin)
        and requested measures have been made available to us.

In this data set, resting potentials $V_R$ ranged from about $-75$ mV to $-50$ mV
            ($-64.83 \pm 1.25$ mV,
            $n = 26$ cells from 3 animals)
        and threshold potentials $V_T$ ranged from about $-60$ mV to $-40$ mV
            ($-51.48 \pm 0.92$ mV,
            $n = 26$).
    The difference between resting and threshold potentials
        $\Delta V_{TR}$ in these recordings (the distance in mV)
            \NEWCHANGE{ranged from $5$ to $22$ mV}
            ($13.34 \pm 0.88$ mV,
            $n = 26$)
        (\autoref{fig:ffwd-biol-data}).
    This analysis shows the range of threshold potentials assumed
        in a biological setting and under physiological conditions
        in the cortex of the awake macaque,
        an animal model that is as close as it gets to the human cortex.

\subsubsection*{Threshold potentials and plasticity in S1 \textit{in vitro} recordings}

Our own \textit{in vitro} recordings
        from the primary sensory cortex (S1) of the mouse \citep{Gill2020-wy}
        allow us to perform more detailed analyses,
        including the analysis of activity parameters that
        alter the threshold potential.
    The data are shown in \autoref{fig:ffwd-biol-data}b.
    We observe that the resting potentials $V_R$ in S1 assume values
        that are similar to those seen in the V1 cortex of awake macaques,
        roughly in the range of $-75$ mV to \NEWCHANGE{$-45$} mV
            \NEWCHANGE{(pre $= -65.50 \pm 0.98$ mV,
                post $= -65.73 \pm 1.25$ mV,
                paired pre-vs-post $t$-stat $= 0.3807$,
                $p = 0.7063$,
                $n = 29$ cells)}.
        \NEWCHANGE{We did not observe significant differences between
            the resting potentials of S1 (combined pre \& post) and V1 data
                ($t$-stat $= -0.5433$,
                $p = 0.5884$)}.
    However, the threshold potentials $V_T$ occupy
        a more positive range than those observed in the V1 recordings;
        we observed a range from about $-50$ mV to \NEWCHANGE{$-30$} mV
            \NEWCHANGE{(pre $= -42.14 \pm 0.48$ mV,
                post $= -42.22 \pm 0.74 $ mV,
                paired $t$-stat $= 0.1311$,
                $p = 0.8966$,
                $n = 29$),
            as well as significant difference between S1 and V1 threshold potentials $V_T$
                ($t$-stat $= 10.3571$,
                $p = 1.4765 \times 10^{-16}$)
            },
        leading to a larger difference between resting and threshold potentials
            \NEWCHANGE{($\Delta V_{TR}$) in S1 than in V1
                (V1 $= 13.34 \pm 0.88$ mV,
                S1 pre $= 23.36 \pm 1.06$ mV,
                S1 post $= 23.51 \pm 1.45$ mV;
                V1-vs-S1 two-sample $t$-stat $= 6.9342$,
                $p = 8.6032 \times 10^{-10}$)}.
    \NEWCHANGE{A caveat of our V1 to S1 comparisons certainly is that
            the values were taken from different recordings conditions
            (\textit{in vivo} for V1,
            \textit{in vitro} -- without an attempt to ``activate'' the slice -- for S1).
        Note, however, that the}
        finding of a larger $\Delta V_{TR}$ corresponds
        to observations made in S1 cortex of intact rats \citep{Brecht2003-vf}.

\begin{figure}[H]
    \centering
    \includegraphics[width=1.2\textwidth,center]{figures/Fig5.pdf}
    \caption{
    \textit{Threshold potentials and plasticity: analysis of biological data sets}.
    (\textbf{a}) Biological data of
            neural spiking threshold ($V_T$),
            resting potentials ($V_R$)
            and their differences  ($\Delta V_{TR} = V_T - V_R$)
        from \textit{in vivo} V1 recordings
            obtained from \cite{Li2020-ej}.
        In each violin plot,
            horizontal lines show the mean,
            while white circles show
                the first quartile, median and third quartile from bottom to top.
    (\textbf{b}) Biological data of $V_T$, $V_R$ and $\Delta V_{TR}$
        from \textit{in vitro} data taken from \cite{Gill2020-wy}
        at initial values ($V^0$)
        and after plasticity induction ($V^f$).
        Colors indicate different plasticity induction protocols:
            \textit{electrical} activation (somatic depolarization or synaptic activation) in blue;
            \textit{cholinergic} group using oxo-m bath application in green;
            and \textit{cholinergic paired} with electrical activation (somatic depolarization) in red.
    (\textbf{c}) The recognition tradeoff (shades of red)
            and recognition true-positive rate (shades of blue)
            as a function of threshold change ($\Delta\theta$)
            for three conditions of selectivity indices ($\alpha_W$)
        are shown as an example of the model calculations
            to compare with the lower and upper threshold potential bounds measured in mouse S1 cortex.
        The lower bound is the average of resting potentials ($n = 23$),
            while the upper bound is the average of threshold potentials
                during the relative refractory period calculated from a new data set
                ($n = 4$, see \autoref{supp:demo-refrac-thres}).
    (\textbf{d}) Relationship between functional change (number of spikes $n_{spk}$)
            and membrane potential change ($V_R, V_T, \Delta V_{TR}$)
            after induction protocols are applied (color legends are same as in panel b).
        Change is denoted as
            $x^{\Delta} = x^{\mathrm{f}} - x^{\mathrm{0}}$,
            as the difference between initial and final values of variable $x$.
        Colored lines are linear fits using data from each induction group,
            while dashed gray lines are linear fits using all the data.
    (\textbf{e})
    Distribution of S1 recordings
        from the three different experimental groups
        according to category membership,
        based on $\Delta V_{TR}^{\Delta}$,
        to either
        ``negative threshold shift''
        or ``non-negative threshold shift''.
        See text and \autoref{supp:s1-data-dV_TR} for more information.
    }
    \label{fig:ffwd-biol-data}
\end{figure}



\NEWCHANGE{Additionally, we also compare the pre- and post-induction
        membrane potential variables ($\Delta V_{TR}, V_T, V_R$)
        within each group separately,
        using the Student paired $t$-tests.
    We observed significant negative threshold $V_T$ changes
        with paired electric and cholinergic activation
            ($-1.96 \pm 0.79$ mV,
            $t$-stat $= 2.4897$,
            $p = 0.0375$,
            $n=9$),
        but not with electric
            ($0.23 \pm 0.72$ mV,
            $t$-stat $= -0.3210$,
            $p = 0.7555$,
            $n = 10$)
        or cholinergic activation alone
            ($1.29 \pm 1.51$ mV,
            $t$-stat $= -0.8549$,
            $p = 0.4148$,
            $n = 10$).
    We did not observe any significant differences within each group for
        the other two membrane potential variables $\Delta V_{TR}, V_R$.
}

Significant negative threshold changes were observed with paired electric and cholinergic activation (-1.96 mV +/- 2.37 SD; n=9; p=0.0375; paired Student's t-test), but not with electric (+0.23 mV +/- 2.28 SD; n=10; p=0.7544) or cholinergic activation alone (+1.29 +/- 4.76 SD; n=10; p=0.4150).

We cannot fit the thresholds observed directly to our model shown in \autoref{fig:ffwd-recog-perf}a,
        but made an attempt to approximate minimal and maximal threshold potential values
        that correspond to the $-0.5$ and $0.5$ edges of the model plots.
    As the spike threshold $V_T$ cannot be lower than the resting potential $V_R$,
        we accepted the average resting potential measured in our recordings
        as an approximation of
        $\theta_{\mathrm{min}} = 0, \Delta \theta_{\mathrm{min}} = -0.5$.
    The value obtained from our recordings
        \NEWCHANGE{($n=29$) is $-65.61$ mV}
        (combined pre and post measurements).

As the theoretically available most positive threshold bound
        $\theta_{\mathrm{max}} = 0, \Delta \theta_{\mathrm{max}} = 0.5$,
        we accepted the membrane potential at which action potentials are evoked
            from a depolarizing pulse when the neuron is in a (relative) refractory state
        \TOREMOVE{(see {\autoref{supp:demo-refrac-thres}})}.
    In a set of new recordings, this value was determined to be $-30.7$ mV
        ($n=4$; \autoref{fig:ffwd-biol-data}c).
    Therefore, the threshold shifts available for optimization of performance in the recognition task
        are limited to a maximal range of about $35$ mV.
    Note that in our set of
        \NEWCHANGE{$29$ recordings},
        the range of ``real'' (measured in the non-refractory state) threshold potentials
        was about
        \NEWCHANGEUNSURE{$10-15$ mV wide}\footnote{Is this the bounds or average of $\Delta V_{TR}$}.
    Although this observation does not exclude the possibility that larger changes can take place,
        in particular with repeated instead of single activation periods,
        it seems likely (and plausible) that under physiological conditions
            the range of assumed threshold potentials
            is smaller than the theoretically available one.

The \textit{in vivo} V1 data set \citep{Li2020-ej} from the Priebe laboratory does not contain plasticity data,
        but our own published recordings from L2/3 pyramidal neurons in S1 mouse cortex \textit{in vitro} do \citep{Gill2020-wy}.
    As reported in \cite{Gill2020-wy},
        three manipulations lead to increases in the measure of excitability used in this work,
        the number of spikes evoked by somatic injection of a depolarizing current test pulse.
    In a total of
        \NEWCHANGE{29  recordings (23 from that study and 6 additional recordings)},
        the experimental manipulations and number of recorded cells per manipulation are as follows:
        electric activation (repeated injection of depolarizing currents or tetanic activation of excitatory synapses; $n=10$);
        bath application of the muscarinic agonist oxo-m ($7 \mu M$; \NEWCHANGE{$n=10$});
        paired oxo-m application with electric stimulation (\NEWCHANGE{$n=9$}).
    All three protocols cause an increase in excitability \citep{Gill2020-wy}
        and therefore qualify as efficient IP protocols \autoref{fig:ffwd-biol-data}d.
    Changes in threshold potential have not been analyzed and reported in \cite{Gill2020-wy},
        but are examined here.
    First, we notice that threshold plasticity ($V_T^{\Delta}, \Delta V_{TR}^{\Delta}$) is bidirectional,
        even though the change in spike number ($n_{spk}^{\Delta}$) is largely unidirectional.
    This means that in some cells the threshold for spike firing is enhanced,
        but once passed, more spikes can be evoked.
    Second, this bidirectional change leads to a widening of the range of threshold potentials
        after IP is triggered (\autoref{fig:ffwd-biol-data}b).
    \NEWCHANGE{Additionally, we observe significant negative association between
            the changes in spike number ($n_{spk}^{\Delta}$) and
            threshold-to-rest distance changes ($\Delta V_{TR}^{\Delta}$)
            ($\rho = -0.41, R^2 = 0.17, p = 0.02753$),
            but not with the changes of the other two membrane potential variables
            ($ p = 0.0996 $ for $V_{T}^{\Delta}$
            and $ p = 0.3547$ for $V_{R}^{\Delta}$).
        However, we do not observe any significant effect of the experimental groups
            on any of the response changes
            ($n_{spk}^{\Delta}, \Delta V_{TR}^{\Delta}, V_{T}^{\Delta}, V_{R}^{\Delta}$)
            under four separate one-way ANOVA tests
            ($F$-stats = $0.5681,1.2481, 2.2697, 0.7772$
            and $p = 0.5735, 0.3037, 0.1234, 0.4701$ respectively);
            nor do we find any significant effect of the experiment groups on
            the multivariate response changes using
            the one-way multivariate ANOVA test
            (\texttt{MATLAB/manova1},
            $d=0$,
            $p[d=0] = 0.5012$,
            $p[d=1] = 0.4722$)
    }.

\TOREMOVE{
Tested with multi-way ANOVA with pairwise interaction model
    with 3 main variables (%
        experimental groups,
        threshold potential change $V_T^{\Delta}$
        and resting potential change $V_R^{\Delta}$)
        \footnote{$\Delta V_{TR}^{\Delta}$ is not included
            as it depends on both $V_T^{\Delta}$, $V_R^{\Delta}$},
    the change in spike number $n_{spk}^{\Delta}$ significantly
        depends on threshold change $V_T^{\Delta}$
        ($F = 5.3, p = 0.0386$).
    This finding hints at differences in the efficacy of these experimental manipulations to initiate threshold potential changes and functionally couple them to the magnitude of the spike output.
}

Our model on the relationship
        between performance in a recognition task and changes in the threshold potential
        suggests that under a wide range of conditions
        task performance benefits from a reduction in the spike threshold.
    The experimental data presented in \cite{Gill2020-wy} allow us to examine
        what type of neuronal activation
        causes a negative threshold shift
        and therefore alter the neuron's coding strategy to one that favors the recognition of known / learned input patterns,
        in particular when the presented pattern is incomplete.
    \NEWCHANGE{As a further test to}
        assess differences in the efficacy to promote negative threshold changes,
        we applied $\chi^2$ test of independence
        for the distribution of threshold potential changes
        triggered by the three types of manipulations used (electric, cholinergic and cholinergic paired with electric),
        along two categories: negative and non-negative threshold changes.
    Here, the category ``negative threshold change'' includes
        \NEWCHANGE{all threshold distance changes, i.e. $\Delta V_{TR}^{\Delta}$}
        in a negative direction regardless of amplitude.
    We defined two exceptions:
        a) the negative threshold change category does not include recordings,
            in which there was a decrease in spike number
                that might offset some effects of the threshold change,
        and b) cells with a ``null'' direction change in threshold potential
            were counted as ``non-negative''.
    Based on the distribution of threshold change values
        \NEWCHANGE{in which their relative changes do not exceed $1\%$},
        \TOREMOVE{({\autoref{supp:s1-data-dV_TR}})},
        we identified \NEWCHANGE{four} recordings,
        in which these \NEWCHANGE{potential} values clustered around zero change.
    To define a margin threshold change that would separate ``null'' from ``not null'' changes,
        we calculated the standard deviation from these recordings
        \NEWCHANGE{(SD = $0.0805$)}
        and set the margin to a value that is three times as large,
        \NEWCHANGE{$0.2414$} mV.
    Thus, a negative threshold change has to pass \NEWCHANGE{$-0.2414$} mV
        to fall into the category ``negative threshold change''.
    With this definition of categories,
        in \NEWCHANGE{$7/9$ recordings}
            with paired cholinergic and electric activation
            a negative threshold change was observed,
        while this was only the case in
            \NEWCHANGE{$3/10$ recordings}
            in which cholinergic activation was applied alone,
        and in $1/10$ recordings in which electric activation was applied alone
        (\autoref{fig:ffwd-biol-data}e).
    The paired cholinergic activation was the only efficient activation pattern to promote a negative threshold change
        \NEWCHANGE{($\chi^2 = 9.6504; p = 0.00802$)}.

