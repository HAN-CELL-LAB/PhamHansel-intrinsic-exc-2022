\section{Key points}

\NEWCHANGE{Intrinsic plasticity may change the action potential threshold
    near the soma of neurons (threshold plasticity),
    thus altering the input-output function
    for all synaptic inputs ``upstream'' of the plasticity location.
A potential problem arising from this shared amplification
    is that it may reduce the ability to discriminate
    between different input patterns.
Here, we assess the performance of an artificial neural network
    in the discrimination of unknown input patterns
    as well as the recognition of known patterns
    subsequent to changes in the spike threshold.
We observe that negative changes in threshold potentials
    do reduce discrimination performance,
    but at the same time improve performance in an object recognition task,
        in particular when patterns are incompletely presented.
Analysis of whole-cell patch-clamp recordings from pyramidal neurons
    in the primary somatosensory cortex (S1) of mice reveals
    that negative threshold changes preferentially result from
        electric stimulation of neurons
        paired with the activation of muscarinic acetylcholine receptors.
}

\section{Abstract}

Activity-dependent changes in membrane excitability
    are observed in neurons across brain areas
    and represent a cell-autonomous form of plasticity (intrinsic plasticity; IP)
    that in itself does not involve alterations in synaptic strength (synaptic plasticity; SP).
Non-homeostatic IP may play an essential role in learning,
    e.g. by changing the action potential threshold near the soma.
A computational problem, however, arises from the implication that
    such amplification does not discriminate between synaptic inputs
    and therefore may reduce the resolution of input representation.
Here, we investigate consequences of IP for the performance of an artificial neural network in
    a) the discrimination of unknown input patterns
    and b) the recognition of known / learned patterns.
While negative changes in threshold potentials in the output layer
    indeed reduce its ability to discriminate patterns,
    they benefit the recognition of known, but incompletely presented patterns.
An analysis of thresholds and IP-induced threshold changes
    in published sets of physiological data obtained
    from whole-cell patch-clamp recordings
    from L2/3 pyramidal neurons in
    a) the primary visual cortex (V1) of awake macaques
    and b) the primary somatosensory cortex (S1) of mice in vitro, respectively,
    reveals a difference between resting and threshold potentials
    of $\sim$ 15 mV for V1
    and $\sim$ 25 mV for S1,
    and a total plasticity range of $\sim$ 10 mV (S1).
The most efficient activity pattern to lower threshold
    is paired cholinergic and electric activation.
Our findings show that threshold reduction
    promotes a shift in neural coding strategies
    from accurate faithful representation to interpretative assignment of input patterns to learned object categories.


\NEWCHANGE{%
    \textbf{Keywords}:
        cholinergic;
        intrinsic plasticity;
        pattern recognition;
        pyramidal neuron;
        spike threshold
}
