\section{Abstract}

Activity-dependent changes in membrane excitability
    are observed in neurons across brain areas
    and represent a cell-autonomous form of plasticity (intrinsic plasticity; IP)
    that in itself does not involve alterations in synaptic strength (synaptic plasticity; SP).
\NEWCHANGE{%
    Intrinsic plasticity may change
        the action potential threshold near the soma of neurons (threshold plasticity),
        thus altering the input-output function
        for all synaptic inputs ``upstream'' of the plasticity location.
    A potential problem arising from this shared amplification
        is that it may reduce the ability
        to discriminate between different input patterns.
    Here, we assess the performance of an artificial neural network in
        the discrimination of unknown input patterns
        as well as
        the recognition of known patterns subsequent to changes in the spike threshold.
    We observe that negative changes in threshold potentials
        do reduce discrimination performance,
        but at the same time improve performance in an object recognition task,
        in particular when patterns are incompletely presented}.
An analysis of thresholds and IP-induced threshold changes
    in published sets of physiological data obtained
    from whole-cell patch-clamp recordings
    from L2/3 pyramidal neurons in
    a) the primary visual cortex (V1) of awake macaques
    and b) the primary somatosensory cortex (S1) of mice in vitro, respectively,
    reveals a difference between resting and threshold potentials
    of $\sim$ 15 mV for V1
    and $\sim$ 25 mV for S1,
    and a total plasticity range of $\sim$ 10 mV (S1).
\NEWCHANGE{%
    Analysis of whole-cell patch-clamp recordings from pyramidal neurons
        in the primary somatosensory cortex (S1) of mice reveals that
        negative threshold changes preferentially result from
            electric stimulation of neurons
            paired with
            the activation of muscarinic acetylcholine receptors}.
Our findings show that threshold reduction
    promotes a shift in neural coding strategies
    from accurate faithful representation to interpretative assignment of input patterns to learned object categories.

\NEWCHANGE{%
    \textbf{Keywords}:
        cholinergic;
        intrinsic plasticity;
        pattern recognition;
        pyramidal neuron;
        spike threshold
}
