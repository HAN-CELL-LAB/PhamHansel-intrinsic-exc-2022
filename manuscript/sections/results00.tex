\subsection{Threshold decrease results in a reduced ability to discriminate distinct input patterns}

It is expected that increased neuronal excitability via a decrease in spike threshold
        results in a ``merging'' of output patterns due to a shared signal amplification
            that takes place ``downstream'' (closer to the soma) of multiple otherwise independent synaptic inputs that target the dendrite.
    This downstream amplification may lead to reduced resolution
        and thus a decrease in the ability to discriminate input patterns.
    To quantitatively describe this effect, we model a two-layer feedforward network (\autoref{fig:discrim})
        and inspect the output discriminability via output pattern distributions
        and pairwise distance for both random (initial) and optimized states (via Euclidean loss, see Methods).

Across a wide range of values of $N_X, N_Y$, we observe that
        the effects of threshold change in the output layer $\Delta \theta$
        on the number of unique output patterns and on output entropy
        in the random initial states
        are generally symmetric (\autoref{supp:discrim-ent-nunq}).
    The optimization process via minimization of the Euclidean loss for output patterns
        results in a small decrease in the output thresholds (see \autoref{supp:demo-discrim}c$_2$),
        creating the asymmetry in the peak values of the output entropy and pattern number
            (as a function of $\Delta \theta$ to shift to the right),
        especially for small $N_Y$
        (\autoref{supp:demo-discrim}b,
        \autoref{supp:discrim-ent-nunq}a$_2$,b$_2$).
    The $\Delta \theta$ for the peak values in these curves tends towards positive change,
        exemplifying how decreased thresholds may harm discriminability.


\begin{figure}[ht]
    \centering
    \includegraphics[width=0.9\textwidth,center]{figures/Fig1.pdf}
    \caption{
    \textit{Output threshold decrease reduces discriminability}.
    (\textbf{a}) Schematic of the model and quantification of discriminability.
        For each $k$, a set of binary inputs $\{X^{(i)}\}$ with the first set of $k$ active units
            (minimal overlapping activation) are considered more ``discriminable''
        if the pairwise distance (either with Jaccard distances $J_d$ or normalized Hamming distances $H_d$)
            of the corresponding outputs $\{Y^{(i)}\}$ is larger.
        $W$ is the weight matrix between $X$ and $Y$, $\theta$ is the output layer threshold
            while $\Delta \theta$ is the applied threshold change.
    (\textbf{b}) Illustration of the way $J_d$ (red) and $H_d$ (blue) change for specific pairs of outputs
        when $\Delta \theta$ changes the excitability of the outputs (green dots).
    (\textbf{c}, \textbf{d}) The mean $J_d$ (red) and $H_d$ (blue) distances and mean output activity (black) are plotted
        as a function of $\Delta \theta$ for different $k$ (shadings)
        of $N_X = 10$ input units and different number of output units $N_Y$.
    These are shown for the best states, see \autoref{supp:discrim-dist-full} for comparison with the initial states.
    }
    \label{fig:discrim}
\end{figure}

Another way to quantify discriminability is to look at the pairwise distances between the binary output patterns
        using known distance measures such as
            the Jaccard distances $J_d$
            or the normalized Hamming distances $H_d$
            (\autoref{fig:discrim}b-d).
    Instead of observing these distances for all input patterns, we choose
        a minimal common input activation number $k$,
        to parameterize input overlap when quantifying output discriminability.
    The relationship between these distances and threshold changes is shown in \autoref{fig:discrim}c,d for the optimized states,
        as well as \autoref{supp:discrim-dist-full} for both the random initial and best states.

Decreases in thresholds, for both cases, increase output activity as expected
        but decrease the binary distances $J_d$ and $H_d$,
        for different values of input pattern with minimal similarity $k$.
    As $k$ increases, the resulting output patterns become less distinguishable and $\Delta \theta < 0$ worsens the outcome.
    On the other hand, when $\Delta \theta > 0$, the effects still hold mainly for Jaccard distances,
        as they are more biased towards activated units.
    The dependence of the normalized Hamming distances on $\Delta \theta$ is nonmonotonic
        (illustrated in \autoref{fig:discrim}b)
        as sparsification of output activity by a large threshold increase would eventually decrease pattern differences.
    Together, these considerations exemplify how detrimental it is for discrimination performance when output thresholds decrease.
