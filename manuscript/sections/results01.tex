\subsection{Single unit supra-threshold activation with incomplete patterns}


\begin{figure}[ht]
    \centering
    \includegraphics[width=1.2\textwidth,center]{figures/Fig2.pdf}
    \caption{
    \textit{Dependence of the number of input combinations resulting in supra-threshold activation ($N = 10$) on the threshold $\theta$}.
    Supra-threshold activation of the output $Y$ means that
        the weights sum of the inputs $X$ is greater than the threshold $\theta$,
        in other words $Y = 1$ if $\sum_i W_i x_i > \theta$.
    (\textbf{a}) and (\textbf{b}) are examples of the weights drawn from
        a uniform distribution (a)
        and when all are assigned the same value (b), respectively,
        before normalization to the sum.
    In each example panel:
        (\textit{i}) shows the sorted input weights from $X \to Y$;
        (\textit{ii}) shows the sorted weighted sum of the corresponding combination
        of the corresponding combinations of the inputs (\textit{iii});
        (\textit{iv}) shows whether corresponding combinations in (\textit{ii,iii}) would result in supra-threshold activation as a function of the threshold $\theta$ (x-axis);
        (\textit{v}) indicates the number of combinations in (\textit{iv}) showing supra-threshold activation, as a function of the threshold $\theta$ (x-axis).
    (\textbf{c}) Mean sorted weights for 50 instantiations for
        the equal-weight scenario (dotted line)
        and for the scenario where weights are initially drawn from a uniform distribution (dashed line).
    (\textbf{d}) Mean number of combinations resulting in supra-threshold activation, as a function of the threshold.
    }
    \label{fig:demo-simple}
\end{figure}

A central hypothesis of this study is that -- in contrast to pattern discrimination -- the recognition of known,
        but incomplete patterns may benefit from enhanced intrinsic excitability.
    To provide a proof-of-principle demonstration
        that a reduction in spike threshold enables neuronal activation with a lower number of synaptic inputs,
        we start with the simple example of
        a binary unit $Y$ activated by a Heaviside step function (\autoref{eq:heavisidefun}).
    Its intrinsic excitability is characterized by the threshold of activation  $\theta$.
    We assume that its activity represents the active state of all of its inputs in $X$ (\autoref{fig:demo-simple}).
    The connectivity $W$ from the input units in X to the output of interest Y could be generated in various ways.
    For simplicity, we choose to either draw them from
        a uniform distribution (\autoref{eq:defunifnorm} and \autoref{fig:demo-simple}a)
        or assign equal values (\autoref{fig:demo-simple}b).
    For the former, on average, the weights would decrease linearly with their order (\autoref{fig:demo-simple}c).
    Either way, the weights are normalized to a sum of 1, for simplified control of the output threshold range.

When ignoring false positives (for now), an incomplete pattern in $X$ would be any pattern where
        at least one unit is inactive
        and the resulting active state of the output signifies the recognition of an input pattern despite such incompleteness.
    As shown in \autoref{fig:demo-simple}a,b (sub-panels \textit{iv, v}),
        when there are not enough input units active
        or without the presence of high-weight inputs,
        low thresholds are required to evoke responses in the output unit.
    Although more inputs or the presence of stronger inputs would allow for more flexibility in the range of required thresholds (sub-panels \textit{v}),
        in the absence of low thresholds the trade-off is that the number of patterns allowing for the recognition becomes limited,
        and that weaker inputs are likely insufficient.
    This might be problematic in cases where weak inputs convey information about relevant object details.
    A solution is lowering the threshold (sub-panels \textit{ii, v}, \autoref{fig:demo-simple}d) of the output.
    In neural network terms, increased excitability
        -- for example resulting from activity-dependent plasticity of membrane excitability --
        would allow for more patterns, even incomplete ones,
        to be sufficient to evoke supra-threshold responses regardless of the arrangement of input weights,
        e.g. few synaptic inputs with high input weights or more with low input weights.

This simple toy example hints at the possible computational benefit of intrinsic excitability
        when recognition is desired, but input components are missing.
    However, this example assumes that all inputs represent the object of interest,
        which is not the case when unnecessary inputs,
        or inputs belonging to another object (category) are present.
    Next, we are going to include irrelevant inputs to examine the possible problems and benefits
        for the recognition of the relevant input pattern.